\section{Sostenibilidad}
\subsection{Autoevaluación}
Desde el primer semestre del grado en ingeniería informática se ha dado especial importancia a
la sostenibilidad como concepto a tener siempre presente. Sin embargo, personalmente no ha sido hasta
la elaboración del trabajo de fin de grado que me he planteado la huella social, económica y 
ambiental de un proyecto como este.

Realizando un ejercicio de autoevaluación me he dado cuenta de que durante el planteamiento del
trabajo no he tenido en consideración algunos criterios para validar un
proyecto sostenible. Concretamente me ha sorprendido la cantidad de indicadores existentes
para evaluar el impacto en las distintas dimensiones de la sostenibilidad y considero que
són una herramienta clave para detectar problemas de sostenibilidad y poder resolverlos.

\subsection{Dimensión económica}
\textbf{¿Has estimado el coste de la realización del proyecto (recursos humanos y materiales)?}

En la sección \ref{budget} se estima el coste de los distintos recursos humanos para cada
tarea y se calculan los costes genéricos, que principalmente són materiales.
Además se define un margen de contingencia y otros costes asociados al riesgo y los obstáculos.

\textbf{¿Cómo se resuelve actualmente el problema que quieres abordar (estado del arte)?
¿En qué mejorará económicamente tu solución a las existentes?}

El método de resolución que se plantea en este trabajo es el estado del arte actual.
Las soluciones existentes se ofrecen como servicios de pago con tarifas poco atractivas
\cite{GoogleTranslatePricing,AWSTranslatePricing,MicrosoftTranslatePricing}. El proyecto pretende
hacer accesible una solución \textit{in-house} para empresas con necesidad de traducir grandes
volumenes de contenido.

\subsection{Dimensión ambiental}
\textbf{¿Has estimado el impacto ambiental que tendrá la realización del proyecto? ¿Te has
planteado minimizar el impacto, por ejemplo, reutilizando recursos?}

El trabajo ejerce un impacto sobre el medio ambiente de diversas formas, algunos de estos
agentes són el consumo eléctrico, agua, transporte, internet, espacio de trabajo y maquinaria.
A pesar de que no se ha estimado el impacto, se han tomado medidas para minimizar
sus efectos sobre el medio:

El uso de ordenadores de alto rendimiento tiene un gran coste ambiental,
desde su fabricación hasta su posterior reciclaje. Es por ello, que la maquinaria
local \hyperref[R06]{R06} se ha utilizado previamente y se seguirá usando después de
finalizar el proyecto. Se pretende reutilizar al máximo este recurso hasta el
momento de su disposición y reciclaje.
En cuanto a los recursos computacionales remotos, se confía en las políticas de sostenibilidad
\cite{awssustainability} que anuncian los distintos proveedores.

Además, mediante el uso de técnicas como el \textit{fine-tunning} y el
\textit{transfer learning}, se pretende minimizar el consumo eléctrico necesario para
entrenar los modelos de traducción.

Por último, el uso de transporte público y la consumición de recursos y servicios sostenibles
reducen mi impacto como individuo y principal fuerza de trabajo del proyecto. 

\textbf{¿Cómo se resuelve actualmente el problema que quieres abordar (estado del arte)?
¿En qué mejorará ambientalmente tu solución a las existentes?}

Actualmente los servicios de traducción automática se alojan en centros de datos y
es difícil estimar su impacto ambiental debido a la poca transparencia de los proveedores
de estos servicios. La solución que plantea este trabajo permitirá realizar un aprovisionamiento
de los recursos computacionales más adecuado a las necesidades de la empresa, posibilitando incluso
su implantación \textit{in-house}. Además, se facilitará el ajuste de los modelos y se
explorarán opciones de mayor rendimiento.
Esto se convertirá en una reducción en el consumo eléctrico y al mismo tiempo dará mejor uso al
\textit{hardware}.


\subsection{Dimensión social}
\textbf{Qué crees que te va a aportar a nivel personal la realización de este proyecto?}

El trabajo de fin de grado es hasta la fecha el proyecto de \textit{machine learning} más
complejo y extenso que he realizado, además tiene un fuerte componente de investigación
así como de desarrollo práctico. Considero que ganaré cierto grado de madurez
en el campo y me permitirá afrontar proyectos de mayor envergadura en el futuro.

\textbf{¿Cómo se resuelve actualmente el problema que quieres abordar (estado del arte)?
¿En qué mejorará socialmente (calidad de vida) tu solución a las existentes?}

Actualmente la mayoria de servicios de traducción automática se limitan a fragmentos y documentos
con poco volumen de texto. Las soluciones para la traducción de textos largos
són escasas y en el caso de documentos suelen ofrecerse
como un servicio de pago difícilmente escalable a nivel empresarial. Además, no es posible
el ajuste de los modelos de traducción.

El proyecto pretende ser una alternativa que podrán utilizar individuos ajenos al \textit{machine learning},
con la capacidad de traducir fragmentos de texto y documentos de distintos formatos mediante
el uso de modelos que representan el estado del arte.
Se permitirá añadir, corregir y ajustar los modelos para la traducción en registros y contextos
específicos.

\textbf{¿Existe una necesidad real del proyecto?}

La traducción es una tarea esencial en el ámbito personal y empresarial. Las prestaciones que se proponen
suponen una mejora sustancial sobre las soluciones existentes y existe una necesidad real en empresas
e individuos con un gran volumen de contenido que necesita ser traducido automáticamente.
 