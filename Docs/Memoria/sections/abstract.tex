\renewcommand{\abstractname}{Abstract}
\begin{abstract}
The transformer architecture \cite{Vaswani2017Jun} currently represents the state of the art in many areas of natural language processing. Moreover, one of its earliest applications was machine translation, and it has demonstrated exceptional performance in this task. However, several companies with translation needs lack the means to implement their own models. In turn, commercial solutions don't allow model adjustment and their economic cost is not scalable in companies that handle large volumes of documents.

In this work, the transformer architecture is investigated and machine translation models are developed using finetuning and transfer learning techniques. Finally, an enterprise web tool has been implemented for non-machine learning users, which allows the translation of text fragments and pdf, docx and txt documents. A correction mechanism has also been added and the service has been designed to be easily deployed in-house by any interested company.
\paragraph{Keywords:} machine translation, transformer architecture, finetuning, transfer learning, tokenización, service orchestration.
\end{abstract}
\renewcommand{\abstractname}{Resumen}
\begin{abstract}
La arquitectura transformer \cite{Vaswani2017Jun} actualmente supone el estado del arte en muchas areas del procesamiento del lenguaje natural. Además, una de sus primeras aplicaciones fue la traducción automática, y ha demostrado un rendimiento excepcional en esta tarea. Sin embargo, algunas empresas con necesidades de traducción, carecen de los medios para la implementación de modelos propios y a su vez, las soluciones comerciales no permiten el ajuste de los modelos y su coste económico no es escalable en empresas que manejan grandes volúmenes de documentos.

En este trabajo se investiga la arquitectura transformer y se desarrollan modelos de traducción automática usando las técnicas de \textit{finetuning} y \textit{transfer learning}. Finalmente, se ha implementado una herramienta web empresarial orientada a usuarios ajenos al aprendizaje automático, que permite la traducción de fragmentos de texto y documentos pdf, docx y txt. También se ha añadido un mecanismo de corrección y se ha diseñado el servicio para ser fácilmente desplegado \textit{in-house} por cualquier empresa interesada.
\paragraph{Palabras clave:} traducción automática, arquitectura transformer, \textit{finetuning}, \textit{transfer learning}, tokenización, \textit{service orchestration}.
\end{abstract}
\renewcommand{\abstractname}{Resum}
\begin{abstract}
L'arquitectura transformer \cite{Vaswani2017Jun} actualment suposa l'estat de l'art a moltes àrees del processament del llenguatge natural. A més, una de seves primeres aplicacions ca ser la traducció automàtica, i ha demostrat un rendiment excepcional en aquesta tasca. Tanmateix, algunes empreses amb necessitat de traducció, manquen dels medis per a la implementació de models propis i així mateix, les solucions comercials no permeten l'ajust dels models i el seu cost econòmic no és escalable en empreses que maneguen grans volums de documents.

En aquest treball s'investiga l'arquitectura transformer i es desenvolupen models de traducció automàtica utilitzant les tècniques de \textit{finetuning} i \textit{transfer learning}. Finalment, s'ha implementat una eina web empresarial orientada a usuaris aliens a l'aprenentatge automàtic, que permet la traducció de fragments de text i documents pdf, docx i txt. També s'ha afegit un mecanisme de correcció i s'ha dissenyat el servei per a ser fàcilment desplegat \textit{in-house} per qualsevol empresa interessada.
\paragraph{Palaules clau:} traducció automática, arquitectura transformer, \textit{finetuning}, \textit{transfer learning}, \textit{tokenization}, \textit{service orchestration}.
\end{abstract}

\chapter*{Agradecimientos}
A mi director Javier Béjar Alonso por solventar todas mis dudas y guiarme a lo largo del desarrollo del proyecto.

Quiero agradecer a mi familia por el apoyo que me han brindado a lo largo de la elaboración del trabajo y al soporte de mis amigos, que me han proporcionado opiniones honestas sobre la calidad de los modelos implementados cuando lo he necesitado.