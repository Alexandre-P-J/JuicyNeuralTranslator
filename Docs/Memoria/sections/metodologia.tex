\chapter{Metodología y rigor}
\section{Descripción de la metodología}\label{methodology}
El desarrollo del trabajo requiere la investigación de arquitecturas y técnicas de \textit{machine learning},
además de una parte práctica que puede demostrar ser difícil debido a los requisitos computacionales
y temporales necesarios.
Es por ello que se debe usar una metodología que permita maximizar el desempeño al mismo tiempo
que facilita la validación del trabajo durante su desarrollo.

La metodología Scrum encaja con las necesidades del proyecto ya que enfoca el desarrollo de
forma iterativa mediante la definición de ciclos de trabajo cortos seguidos de sesiones
de validación que permitirán el ajuste de la planificación y la asignación de nuevas tareas.

Se realizarán reuniones cada 2 semanas con el director del trabajo y se comprobará
el estado del proyecto. El final de la reunión se dedicará a la definición y asignación de
nuevas tareas con el objetivo de que se realicen durante la siguiente iteración.
Además, en el transcurso de los ciclos de trabajo también se obtendrá información
que permitirá ajustar y mejorar las iteraciones futuras.


\section{Herramientas de seguimiento}\label{methodtools}
La validación del progreso del proyecto se realizará mediante reuniones con el director
tal como se ha descrito en el apartado \ref{methodology}. Principalmente se usará
\textit{Google meet}, pero si la situación lo permite, algunas reuniones podrian ser presenciales.
También se usarán correos electrónicos como vía de contacto con el director y un calendario
que ayudará en el seguimiento de la planificación.  

Se usará \textit{git} como utilidad de versionado del código y \textit{Github} para permitir
la distribución del proyecto en la nube. El código se organizará en dos ramas: \textit{development} y \textit{master}. La primera
será una rama de trabajo con contenido en desarrollo mientras que \textit{master} contendrá
una versión más estable del proyecto. El progreso realizado en la rama \textit{development} se
trasladará a la rama \textit{master} segun sea necesario.

\section{Desviaciones de la metodología inicial}\label{desvmethod}
Durante el transcurso de la elaboración del trabajo no se han realizado modificaciones a la metodología pero si se ha validado su efectividad.
Tal como se especificó en el apartado \ref{methodology}, se han realizado reuniones cada dos semanas que han servido para encaminar el proyecto bajo la supervisión del director. Además se han realizado demos del proyecto y de los resultados obtenidos siempre que ha sido posible, siguiendo las directivas clásicas de la metodología Scrum.

Las herramientas definidas en el apartado \ref{methodtools} han sido acertadas ya que el desarrollo del proyecto ha sido fluido. El único cambio ha sido la sustitución del servicio \textit{Github} por \textit{Gitlab} debido a que la quota gratuita de almacenamiento de datos binarios es superior en este último servicio.
