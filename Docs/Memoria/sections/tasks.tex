% Les tasques s’han identificat i descrit de manera excellent i es constata que:
% a) El nivell de detall es fa a nivell tasca (i no a nivell fases o grups de 
%    tasques) i les explicacions son completes;
% b) Inclou estimacions en hores per cada tasca;
% c) Es descriu la seva seqüència lògica i s’han explicitat formalment les
%    dependències de precedència;
% d) inclou els recursos (humans i materials) que seran necessaris;
% e) inclou una taula resum de les tasques.

\chapter{Planificación}\label{desctasks}
La carga académica de este trabajo corresponde a 18 créditos. La normativa
de la FIB estima una equivalencia de 30 horas por crédito, que según mi experiencia
personal, es una estimación a la alza.

Se ha decidido realizar la lectura del trabajo en el turno de enero por ser la más
temprana posible y se calcula una duración de 580 horas, 40 horas más que la
estimación de la normativa. El trabajo empieza el 13 de septiembre
de 2021 y continua hasta el 17 de enero de 2022, la semana previa al período de lectura.

La dedicación diara promedio es de 4.57 o 4 horas y 34 minutos, cuya distribución dependerá
de la carga de trabajo y examenes de otras asignaturas.

\section{Definición de las tareas}
Se enumeran a continuación las tareas junto con una descripción. En \ref{gantt} se muestra la planificación teniendo
en cuenta las dependencias y paralelismos entre tareas.
\paragraph{T01.\quad Contextualizar y definir el alcance:}\label{T01}
Decidir el contexto del proyecto, su alcance y la metodologia a seguir con ayuda
del director.
\paragraph{T02.\quad Planificar el trabajo:}\label{T02}
Describir las tareas a realizar, planificar el desarrollo siguiendo la
metodología elegida y formalizar los riesgos y obstáculos.
\paragraph{T03.\quad Estimar el presupuesto y analizar la sostenibilidad:}\label{T03}
Identificar y estimar los costes además de los mecanismos de control para controlar las
desviaciones. También se deben autoevaluar la sostenibilidad y sus dimensiones economica,
ambiental y social.
\paragraph{T04.\quad Reuniones con el director:}\label{T04}
Se efectuará una reunión con el director cada dos semanas en la que se validará el trabajo
y se asignarán nuevas tareas.
\paragraph{T05.\quad Revisión y edición del documento final:}\label{T05}
Integrar todas las partes y efectuar una revisión exhaustiva del documento. Seguidamente
editar y rectificar.
\paragraph{T06.\quad Investigar la arquitectura transformer:}\label{T06}
Investigar la arquitectura transformer original y los modelos pre-entrenados que se van
a usar durante el proyecto.
\paragraph{T07.\quad Investigar \textit{fine-tuning} y \textit{transfer learning}:}\label{T07}
Investigar ambas técnicas en el contexto del trabajo y consultar estudios previos sobre
la aplicación de estas técnicas en la traducción automática.
\paragraph{T08.\quad Aprender sobre el tratamiento de datos en NLP:}\label{T08}
Es imprescindible entender las técnicas de adquisición y procesamiento de datos. Esta será
una parte importante del \textit{pipeline} y debe hacerse de forma rigurosa.
\paragraph{T09.\quad Estudiar y seleccionar las métricas para la valoración de modelos:}\label{T09}
Se deben decidir que métricas se usarán para valorar la bondad de los modelos de traducción.
Un buen entendimiento de las métricas ayudará durante todo el proceso y permitirá obtener
buenos resultados.
\paragraph{T10.\quad Investigar arquitecturas más eficientes de transformer:}\label{T10}
Estudiar arquitecturas más recientes de transformer y contemplar la posibilidad de
implementar un modelo de mayor rendimiento.
\paragraph{T11.\quad Investigar la técnica model \textit{distillation}:}\label{T11}
Si se decide implementar un transformer más eficiente sería ideal usar esta técnica para
reducir el tiempo de entrenamiento necesario.
\paragraph{T12.\quad Implementar el \textit{pipeline} de \textit{fine-tunning}:}\label{T12}
Programar todo lo necesario para entrenar un modelo pre-entrenado y validar su calidad.
\paragraph{T13.\quad \textit{Fine-tunning} de un modelo:}\label{T13}
Aplicar \textit{fine-tunning} a un modelo pre-entrenado usando el \textit{pipeline} programado
anteriormente con un corpus que permita ajustar el modelo de traducción a un contexto del lenguaje
más específico.
\paragraph{T14.\quad Implementar el \textit{pipeline} de \textit{transfer learning}:}\label{T14}
Programar todo lo necesario para realizar \textit{transfer learning} y validar los resultados.
\paragraph{T15.\quad \textit{Transfer learning} a un modelo:}\label{T15}
Aplicar esta técnica sobre un modelo pre-entrenado para que sea capaz de traducir otro
par de lenguas.
\paragraph{T16.\quad Entrenar un modelo más eficiente de transformer mediante \textit{model distillation}:}\label{T16}
La tarea consiste en implementar una versión más eficiente de transformer y aplicar
\textit{model distillation} con el objetivo de replicar el funcionamiento de un modelo pre-entrenado
en uno de mayor rendimiento.
\paragraph{T17.\quad Implementar una web con \textit{placeholders}:}\label{T17}
Programar una versión preliminar con \textit{placeholders} de la aplicación web.
\paragraph{T18.\quad Implementar la traducción de fragmentos de texto:}\label{T18}
La aplicación web debe ser capaz de traducir fragmentos de texto.
\paragraph{T19.\quad Implementar un mecanismo de corrección:}\label{T19}
Las traducciones de fragmentos de texto deberán tener un mecanismo de corrección de errores.
Se almacenarán las correcciones en una base de datos para un posterior
refinamiento de los modelos.
\paragraph{T20.\quad Implementar traducción de archivos de texto plano:}\label{T20}
La aplicación web permitirá subir archivos y traducir el formato txt.
\paragraph{T21.\quad Implementar traducción de archivos de texto enriquecido:}\label{T21}
Dar soporte a archivos de texto enriquecido de tamaño arbitrario mediante la interfaz
de subida de archivos.
\paragraph{T22.\quad Implementar soporte parcial de traducción de archivos pdf:}\label{T22}
Dar soporte parcial a archivos pdf de tamaño arbitrario mediante la interfaz
de subida de archivos.
\paragraph{T23.\quad Pulir el proyecto y prepararlo para su uso empresarial \textit{in-house}:}\label{T23}
Documentar y pulir la estructura del proyecto para facilitar su implementación \textit{in-house}
en una empresa arbitraria.
\paragraph{T24.\quad Redactar la memoria del trabajo:}\label{T24}
Redactar la memoria de la investigación y desarrollo llevado a cabo.
\paragraph{T25.\quad Preparar la defensa del trabajo:}\label{T25}
Componer y practicar una presentación para defender frente al tribunal del trabajo.


\section{Recursos}
\subsection{Materiales y software}
\paragraph{R01.\quad Latex y Markdown:}\label{R01} La memoria se redacta en latex y la documentación en Markdown.
\paragraph{R02.\quad Thunderbird y Google Meet:}\label{R02} Thunderbird permite gestionar
correos electrónicos y calendarios. Google Meet facilitará realizar reuniones en linea.
\paragraph{R03.\quad Git con Github:}\label{R03} Git es un software de control de versiones que junto con Github
permitirá almacenar el código en la nube.
\paragraph{R04.\quad IDE:}\label{R04} El proyecto usará una IDE para facilitar el desarrollo.
\paragraph{R05.\quad Python y librerias diversas:}\label{R05} El lenguaje de programación principal será python y 
se usará Pytorch y otras librerias de visualización, web, tokenización y lectura de texto enriquecido entre otras.
\paragraph{R06.\quad Hardware local:}\label{R06} Ordenador de mesa con procesador i7 4790K, 16GB ram y tarjeta gráfica
Nvidia 980Ti.
\paragraph{R07.\quad Hardware remoto:}\label{R07} Google colab y Kaggle kernel son opciones viables para llevar a cabo el entrenamiento
de modelos que requieren más memoria de la que dispone mi tarjeta gráfica.
Si fuera necesario se contrataría un servicio de pago como AWS, Linode, Azure o Google
cloud entre otras.

\subsection{Humanos}
\paragraph{R08.\quad Investigador:}\label{R08}
El principal recurso humano es el investigador, Alexandre Pérez Josende,
encargado de estudiar las distintas tecnologias y técnicas que empleará en la
implementación de una herramienta web de traducción.

\paragraph{R09.\quad Director:}\label{R09}
El director Javier Béjar Alonso guiará al investigador en su tarea.

\newgeometry{a4paper,left=1in,right=1in,top=1in,bottom=1in,nohead}
\begin{landscape}
    \begin{table}
    \section{Dependencias y estimación temporal}\label{estimaciontemporal}
    \centering
    \begin{tabular}{ | c | l | c | c | c | }
        \hline
        Id. & Tarea & Tiempo & Dependencias & Recursos \\
        \hline
        \hyperref[T01]{T01} & Contextualizar y definir el alcance & 15h & - & \hyperref[R01]{R01}, \hyperref[R06]{R06}, \hyperref[R08]{R08} \\
        \hline
        \hyperref[T02]{T02} & Planificar el trabajo & 5h & - & \hyperref[R01]{R01}, \hyperref[R06]{R06}, \hyperref[R08]{R08} \\
        \hline
        \hyperref[T03]{T03} & Estimar el presupuesto y analizar la sostenibilidad & 10h & - & \hyperref[R01]{R01}, \hyperref[R06]{R06}, \hyperref[R08]{R08} \\
        \hline
        \hyperref[T04]{T04} & Reuniones con el director & 5h & - & \hyperref[R02]{R02}, \hyperref[R06]{R06}, \hyperref[R08]{R08}, \hyperref[R09]{R09} \\
        \hline
        \hyperref[T05]{T05} & Revisión y edición del documento final & 5h & - & \hyperref[R01]{R01}, \hyperref[R06]{R06}, \hyperref[R08]{R08} \\
        \hline
        \hyperref[T06]{T06} & Investigar la arquitectura transformer & 15h & - & \hyperref[R06]{R06}, \hyperref[R08]{R08} \\
        \hline
        \hyperref[T07]{T07} & Investigar \textit{fine-tuning} y \textit{transfer learning} & 10h & - & \hyperref[R06]{R06}, \hyperref[R08]{R08} \\
        \hline
        \hyperref[T08]{T08} & Aprender sobre el tratamiento de datos en NLP & 10h & - & \hyperref[R06]{R06}, \hyperref[R08]{R08} \\
        \hline
        \hyperref[T09]{T09} & Estudiar y seleccionar las métricas para la valoración de
        modelos & 5h & \hyperref[T06]{T06} & \hyperref[R06]{R06}, \hyperref[R08]{R08} \\
        \hline
        \hyperref[T10]{T10} & Investigar arquitecturas más eficientes de transformer & 10h & \hyperref[T06]{T06}, \hyperref[T09]{T09} & \hyperref[R06]{R06}, \hyperref[R08]{R08} \\
        \hline
        \hyperref[T11]{T11} & Investigar la técnica \textit{model distillation} & 5h & - & \hyperref[R06]{R06}, \hyperref[R08]{R08} \\
        \hline
        \hyperref[T12]{T12} & Implementar el pipeline de \textit{fine-tunning} & 10h & \hyperref[T06]{T06}, \hyperref[T07]{T07} & \hyperref[R03]{R03}, \hyperref[R04]{R04}, \hyperref[R05]{R05}, \hyperref[R06]{R06}, \hyperref[R08]{R08} \\
        \hline
        \hyperref[T13]{T13} & \textit{Fine-tunning} de un modelo & 100h & \hyperref[T06]{T06}, \hyperref[T07]{T07}, \hyperref[T12]{T12} & \hyperref[R03]{R03}, \hyperref[R04]{R04}, \hyperref[R05]{R05}, \hyperref[R06]{R06}, \hyperref[R07]{R07}, \hyperref[R08]{R08} \\
        \hline
        \hyperref[T14]{T14} & Implementar el pipeline de \textit{transfer learning} & 10h & \hyperref[T06]{T06}, \hyperref[T07]{T07} & \hyperref[R03]{R03}, \hyperref[R04]{R04}, \hyperref[R05]{R05}, \hyperref[R06]{R06}, \hyperref[R08]{R08} \\
        \hline
        \hyperref[T15]{T15} & \textit{Transfer learning} a un modelo & 100h & \hyperref[T06]{T06}, \hyperref[T07]{T07}, \hyperref[T14]{T14} & \hyperref[R03]{R03}, \hyperref[R04]{R04}, \hyperref[R05]{R05}, \hyperref[R06]{R06}, \hyperref[R07]{R07}, \hyperref[R08]{R08} \\
        \hline
        \hyperref[T16]{T16} & Entrenar un modelo más eficiente de transformer & 120h & \hyperref[T06]{T06}, \hyperref[T11]{T11} & \hyperref[R03]{R03}, \hyperref[R04]{R04}, \hyperref[R05]{R05}, \hyperref[R06]{R06}, \hyperref[R07]{R07}, \hyperref[R08]{R08} \\
        \hline
        \hyperref[T17]{T17} & Implementar una web con placeholders & 5h & - & \hyperref[R03]{R03}, \hyperref[R04]{R04}, \hyperref[R05]{R05}, \hyperref[R06]{R06}, \hyperref[R08]{R08} \\
        \hline
        \hyperref[T18]{T18} & Implementar la traducción de fragmentos de texto & 10h & \hyperref[T17]{T17} & \hyperref[R03]{R03}, \hyperref[R04]{R04}, \hyperref[R05]{R05}, \hyperref[R06]{R06}, \hyperref[R08]{R08} \\
        \hline
        \hyperref[T19]{T19} & Implementar un mecanismo de corrección & 15h & \hyperref[T17]{T17}, \hyperref[T18]{T18} & \hyperref[R03]{R03}, \hyperref[R04]{R04}, \hyperref[R05]{R05}, \hyperref[R06]{R06}, \hyperref[R08]{R08} \\
        \hline
        \hyperref[T20]{T20} & Implementar traducción de archivos de texto plano & 10h & \hyperref[T17]{T17} & \hyperref[R03]{R03}, \hyperref[R04]{R04}, \hyperref[R05]{R05}, \hyperref[R06]{R06}, \hyperref[R08]{R08} \\
        \hline
        \hyperref[T21]{T21} & Implementar traducción de archivos de texto enriquecido & 15h & \hyperref[T17]{T17} & \hyperref[R03]{R03}, \hyperref[R04]{R04}, \hyperref[R05]{R05}, \hyperref[R06]{R06}, \hyperref[R08]{R08} \\
        \hline
        \hyperref[T22]{T22} & Implementar soporte parcial de traducción de archivos pdf & 15h & \hyperref[T17]{T17} & \hyperref[R03]{R03}, \hyperref[R04]{R04}, \hyperref[R05]{R05}, \hyperref[R06]{R06}, \hyperref[R08]{R08} \\
        \hline
        \hyperref[T23]{T23} & Pulir el proyecto para su uso empresarial \textit{in-house} & 5h & \hyperref[T18]{T18}, \hyperref[T19]{T19}, \hyperref[T20]{T21}, \hyperref[T22]{T22} & \hyperref[R01]{R01}, \hyperref[R03]{R03}, \hyperref[R04]{R04}, \hyperref[R05]{R05}, \hyperref[R06]{R06}, \hyperref[R08]{R08} \\
        \hline
        \hyperref[T24]{T24} & Redactar la memoria del trabajo & 60h & - & \hyperref[R01]{R01}, \hyperref[R06]{R06}, \hyperref[R08]{R08} \\
        \hline
        \hyperref[T25]{T25} & Preparar la defensa del trabajo & 10h & - & \hyperref[R01]{R01}, \hyperref[R06]{R06}, \hyperref[R08]{R08} \\
        \hline
    \end{tabular}
    \caption{Tiempo, dependencias y recursos por tarea. [Elaboración propia]}
    \end{table}
\end{landscape}
\restoregeometry\clearpage