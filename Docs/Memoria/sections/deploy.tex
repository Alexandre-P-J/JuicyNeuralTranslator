\chapter{Implantación de modelos a producción}\label{deployprod}
\section{Gestión de recursos}\label{gestionrecursosimplant}
% Definir, avaluar i seleccionar plataformes de desenvolupament i producció hardware i software per al desenvolupament d'aplicacions i serveis informàtics de diversa complexitat
La herramienta de traducción web que se implementa en este trabajo requiere diversos recursos hardware y software para funcionar adecuadamente. Se espera que el usuario pueda traducir fragmentos de texto o documentos en una interfaz accesible, rápida y simple de navegar, además también se deben poder enviar correcciones.

Un primer diseño podría consistir en un único servidor para la aplicación web y una base de datos. Sin embargo, aunque el servicio podría funcionar con pocos usuarios, no escalaría correctamente ya que las necesidades de un servidor web son distintas a las que requiere un servidor de inferencia para la ejecución de los modelos de traducción. La base de datos y el almacenaje de ficheros también requieren hardware específico y pretender que una sola máquina abarque todos los roles no es una opción realista.

En su lugar, es conveniente desacoplar cada una de las partes que componen el servicio para ajustar el hardware a las necesidades específicas y así abaratar coste al mismo tiempo que se mejora el servicio. Para la elaboración del proyecto se ha decidido separar el servidor web del almacenamiento de ficheros, la base de datos, el procesamiento de documentos y el cómputo de traducciones.

Se ha desarrollado el proyecto usando Docker para la definición de cada componente y Docker Compose para orquestrar la ejecución en local. Una configuración empresarial podría usar las imágenes de Docker que se han implementado directamente en un servicio de cómputo web como AWS o las alternativas de Google o Microsoft, eligiendo el hardware y las tarifas que más se ajusten.

\section{Arquitectura del servicio web}
El diagrama \ref{webdiagram} muestra todas las partes que componen la arquitecta del servicio. Los distintos nodos se enlazan mediante flechas que simbolizan la comunicación o transmisión de datos.
\begin{figure}[H]
    \centering
    \includegraphics[width=300pt]{./img/arquitecturaweb.png}
    \caption{Arquitectura del servicio de traducción [Elaboración propia]}\label{webdiagram}
\end{figure}
Los apartados a continuación detallan cada uno de los componentes y su relación con el resto del servicio.

\subsection{Gestor de mensajes}
El gestor de mensajes es un servidor implementado con Redis para la comunicación y encolado de tareas entre los distintos nodos de la arquitectura. Una característica que no se aprecia en el diagrama \ref{webdiagram}, es la capacidad de escalar la arquitectura horizontalmente incrementando el número de nodos de procesamiento de documentos, computo de traducciones o instancias del portal web. Sin embargo, el diseño se planeó específicamente para habilitar este tipo de comportamiento para poder afrontar distintos niveles de carga dinámicamente.

Esto no sería posible sin el sistema de comunicación que provee un gestor de mensajes como Redis, cuyo objetivo es habilitar la transmisión de tareas desde cualquier instancia del portal web hasta alguno de los nodos de procesamiento o computo. Además, el gestor tiene un funcionamiento asíncrono que permite recuperar los resultados de las tareas sin necesidad de bloquear los distintos componentes de la arquitectura, maximizando así el uso eficiente de los recursos.

\subsection{Instancia de cómputo}
Las instancias de cómputo són los nodos encargados de la ejecución de modelos de traducción como los que se han implementado en este trabajo. 
Estas instancias obtienen tareas de traducción gracias al gestor de mensajes y su objetivo es devolver resultados lo antes posible para enviar la tarea resuelta al gestor.

En el proyecto se usa un sistema que usa inferencia mediante CPU desde Python, ya que es una opción económica y común. Sin embargo, existen soluciones industriales que permiten el uso de GPUs y FPGAs de forma muy efectiva para abarcar un gran volumen de procesamiento con poca latencia. En este trabajo no se exploran estas alternativas, pero el diseño de la arquitectura permite el uso de cualquier forma de procesamiento siempre y cuando se use el gestor de mensajes como interfaz de comunicación.

Internamente cada tarea de traducción dirigida a las instancias de cómputo provee una lista con uno o más textos a traducir, también las etiquetas de origen y destino que especifican los idiomas o cualquier otro indicador para seleccionar el modelo apropiado y realizar la traducción.
La tokenización y cualquier otro comportamiento depende totalmente del modelo que debe ejecutarse, es por ello que cada modelo hereda de la clase \textit{Model} tal como se detalla en el apartado \ref{funcextensibility} y es libre de usar cualquier técnica de segmentación.

En los modelos registrados por el servicio web, se usa la librería Spacy para la segmentación de textos largos en parágrafos mediante reglas independientes al lenguaje usado. De esta forma, es posible traducir textos muy largos en uno o más \textit{batches}. También se reduce el tamaño de las secuencias limitando potencialmente las dependencias a las que un modelo puede atender, sin embargo, empíricamente suele dar mejores resultados por que se conserva mejor la sintaxis original. Tampoco se han observado efectos negativos posiblemente debido a que la traducción necesita dependencias para contextualizar pero rara vez requiere dependencias de otros parágrafos. Si no se usa la segmentación por parágrafos los modelos tienden a convertir oraciones separadas por punto a oraciones yuxtapuestas posiblemente debido al dominio de los datos de entrenamiento.

\subsection{Procesador de documentos}
Las instancias de procesamiento de documentos se encargan de la lectura de documentos procedentes del almacén de ficheros y el procesamiento de los distintos formatos para la extracción del texto a traducir. La extracción de datos en el formato pdf es particularmente difícil debido a que el texto no se almacena como cadenas de caracteres, en su lugar, los pdf contienen información vectorial de los símbolos pero no la correspondencia a los caracteres que representan. Para extraer el texto es necesario realizar operaciones costosas como el \textit{OCR} u otros algoritmos que no garantizan una buena extracción de datos.

El servicio es capaz de procesar documentos con formatos pdf, docx y txt usando distintas librerías de Python, posteriormente agrupa los textos extraídos y genera tareas de traducción que se lanzan al gestor de mensajes para su traducción. Finalmente, obtenidas las traducciones, el procesador de documentos genera un nuevo archivo con los contenidos actualizados, lo guarda en el almacén de ficheros y finaliza la tarea de procesamiento devolviendo el identificador del fichero traducido.

\subsection{Almacén de ficheros}
El almacén de ficheros es el componente encargado de la gestión de ficheros. Almacena los documentos subidos al servicio web para ser traducidos y también gestiona los ficheros ya traducidos y disponibles para la descarga.
Este nodo es accedido por el portal web para almacenar los documentos subidos al servicio y también se usa para la descarga de los ficheros traducidos. Además, internamente se asignan identificadores a los archivos subidos que permiten al procesador de documentos la lectura y el posterior guardado del documento traducido.
La implementación consiste en una única carpeta cuyo contenido podría estar alojado en la nube.

\subsection{Base de datos}
La base de datos se usa únicamente para el registro de correcciones proporcionadas por los usuarios. Se utiliza un servidor de PostgreSQL con dos tablas. La primera tabla almacena información sobre el lenguaje usado para la traducción, su llave primaria es \textit{id} y contiene otra columna no nula y única para el nombre del lenguaje.

La segunda tabla almacena las traducciones de los clientes mediante seis columnas. La columna \textit{id} es la llave primaria, \textit{from\_lang} es una llave foránea que identifica el lenguaje original, \textit{to\_lang} también es una llave foránea pero esta identifica el lenguaje de traducción. La columna \textit{text} contiene el texto original, \textit{translation} corresponde al texto traducido por el servicio y finalmente \textit{translation\_correction} es la corrección efectuada por el usuario. Dado que cualquier nombre es válido para representar un lenguaje, los valores ``Español'' y ``Inglés finetuned 02'' son valores válidos para las columnas \textit{from\_lang} y \textit{to\_lang} e identifican el modelo usado para la traducción.

En el trabajo no se implementa ningún mecanismo de \textit{online learning}, sin embargo, la base de datos permite el \textit{finetuning} de los modelos existentes con las correcciones ofrecidas por los clientes del servicio.

\subsection{Portal web}
El portal web se ha implementado en Python con Flask y se encarga del abastecimiento de contenido Html, Css y Javascript a los navegadores de los clientes del servicio. Además, establece una conexión a la base de datos, al almacén de ficheros y al gestor de tareas donde se lanzan las tareas de traducción.

Se ha diseñado para funcionar sin necesidad de mantener información de estado y eso permite la ejecución de múltiples servidores al mismo tiempo en una o más máquinas para usar mejor los recursos de hardware y abastecer a más usuarios. La información de estado, como por ejemplo el identificador de la tarea de traducción pendiente de ser procesada, la gestiona el propio cliente desde el código Javascript que se ejecuta en su navegador. Cuando un cliente lanza una orden de traducción, su navegador pide periódicamente actualizaciones sobre el estado de la tarea hasta que puede mostrarse en la pantalla del usuario.

\section{Funcionalidades}
\subsection{Traducción de fragmentos de texto}
\begin{figure}[H]
    \centering
    \includegraphics[width=410pt]{./img/webfragtext.png}
    \caption{Traducción de fragmentos de texto en la web [Elaboración propia]}\label{webfrag}
\end{figure}

La figura \ref{webfrag} muestra una captura donde se aprecia el portal web. Se observan dos cajas de texto configuradas mediante el desplegable superior para traducir catalán-inglés con el modelo implementado en este trabajo.
Cuando el usuario modifica el texto en la caja de la izquierda se espera medio segundo desde la última edición y posteriormente se pide al servidor web la traducción. Internamente el traductor web crea una tarea destinada a ser ejecutada por cualquiera de las instancias de cómputo. Dada la etiqueta de origen, en este caso ``Catalan (Standard)'', y la etiqueta de destino ``English (Standard) [Transfer from es-en to ca-en]'', se selecciona el modelo y se realiza la traducción.
Periódicamente, el navegador del cliente consulta el estado de la tarea y una vez completada muestra el resultado en la caja de la derecha. La implementación utiliza AJAX para habilitar la traducción sin requerir el uso de botones u otros mecanismos.

\subsection{Corrección y mejora de los modelos}
\begin{figure}[H]
    \centering
    \includegraphics[width=410pt]{./img/webcorrect.png}
    \caption{Envío de correcciones en el servicio web [Elaboración propia]}\label{webcorrect}
\end{figure}
El mecanismo para enviar correcciones es muy simple, desde el portal web únicamente se debe editar el texto traducido. Hecho esto, aparecerá un botón inferior visible en la figura \ref{webcorrect} que al apretarse enviará la corrección y desaparecerá.
Internamente, al presionar el botón de envío, el servidor web recibe los datos de la petición y posteriormente ejecuta una inserción en la base de datos PostgreSQL de forma asíncrona.
Las correcciones luego pueden ser usadas para el \textit{finetuning} de modelos.

\subsection{Traducción de documentos}
Otra de las funcionalidades principales es la traducción de ficheros pdf, docx o txt completos. Esta operación se puede realizar seleccionado la traducción de ficheros tal como se muestra en la figura \ref{webfile1}.
\begin{figure}[H]
    \centering
    \includegraphics[width=410pt]{./img/webfile1.png}
    \caption{Arquitectura del servicio de traducción [Elaboración propia]}\label{webfile1}
\end{figure}

La figura \ref{webfile1} muestra el resultado al cambiar el modo de traducción de fragmentos de texto a subida de documentos. Esta interfaz reemplaza la caja izquierda por un panel con un único botón para la subida de archivos.
Una vez presionado el botón, se abrirá un menú del sistema donde se podrá seleccionar un documento y subirlo al servicio.

\begin{figure}[H]
    \centering
    \includegraphics[width=410pt]{./img/webfile2.png}
    \caption{Arquitectura del servicio de traducción [Elaboración propia]}\label{webfile2}
\end{figure}

En la figura \ref{webfile2} se muestra la interfaz después de haber seleccionado un documento. Se muestra un \textit{spinner} indicando que la tarea de traducción esta en curso. Internamente el servidor web recibirá el documento y lo almacenará en el almacén de archivos, luego creará una tarea de procesamiento de documento que contendrá el identificador del fichero subido. En cualquiera de las instancias de procesamiento de documentos, se procesará el documento subido y se emitirán nuevas tareas de traducción agrupando los distintos fragmentos de texto extraídos del documento.

Una vez procesadas las tareas de traducción por las instancias de cómputo, el procesador de documentos compondrá el nuevo archivo traducido y lo almacenará en el almacén de archivos. Finalmente, se retornará el nuevo identificador del documento disponible en el almacén, y cuando el navegador del usuario pida una actualización de estado para la tarea, si ha finalizado, el servidor web devolverá un enlace hacia el documento para habilitar la descarga.

\begin{figure}[H]
    \centering
    \includegraphics[width=410pt]{./img/webfile3.png}
    \caption{Arquitectura del servicio de traducción [Elaboración propia]}\label{webfile3}
\end{figure}

La figura \ref{webfile3} muestra la interfaz web una vez finalizada la traducción del documento subido. El navegador ha recibido el enlace de descarga mediante una petición AJAX y ha generado un nuevo botón para la descarga.

\subsection{Extensibilidad}\label{funcextensibility}
Todo el proyecto se ha desarrollado con el principio de ser usable de forma empresarial, concretamente, el uso de Docker y la separación de los distintos componentes de la arquitectura permiten incorporar el servicio para un uso profesional. Además, el proyecto es fácilmente extensible y es muy simple añadir nuevos modelos.

Para añadir un nuevo modelo únicamente es necesario añadir un fichero de Python en la carpeta \textit{models} del nodo de cómputo, este script se ejecutará durante el arranque de cada instancia de cómputo y puede usarse para registrar un nuevo modelo de traducción. Para ello, solo es necesario añadir una clase que herede del objeto \textit{Model} e implementar las funciones \textit{get\_source\_langs}, \textit{get\_target\_langs} y \textit{batch\_translate}.

La función \textit{get\_source\_langs} debe retornar el conjunto de etiquetas origen que el modelo es capaz de traducir, un ejemplo sería el conjunto que contiene la etiqueta ``Español de España'', \textit{get\_target\_langs} debe retornar el conjunto de etiquetas destino que el modelo puede traducir, por ejemplo: ``Catalán'' y ``Catalán \textit{Finetune} 01''.
La unión de las etiquetas origen de todos los modelos serán las opciones que aparezcan en el selector izquierdo del portal web, mientras que la unión de las etiquetas destino serán las que aparezcan en el selector derecho.
La función \textit{batch\_translate} recibe una lista de textos, una etiqueta de origen y otra de destino, y debe ser capaz de traducir todos los textos de la lista y retornarlos como otra lista en el mismo orden. Generalmente las etiquetas que se pasan a esta última función, únicamente se usan en modelos multiling{\"u}e para insertar tokens que condicionan al modelo a traducir en los lenguajes o etiquetas especificados.