\chapter{Gestión económica}\label{budget}
\section{Presupuesto}
\subsection{Costes de personal}
En esta sección se van a calcular los costes de personal asociados a cada una de las tareas
definidas en \ref{desctasks}. A pesar de que todos los roles del personal se reparten entre
el director y el autor de este trabajo, los costes se calcularán para cada una de las cuatro
posiciones: jefe de proyecto, científico de datos, programador y \textit{tester}.
El jefe del proyecto es el responsable de de la planificación y la ejecución del proyecto,
el científico de datos estudiará e implementará los distintos modelos y técnicas de
\textit{machine learning} necesarios, el programador diseñará y programará una herramienta
que usará los modelos de traducción y el \textit{tester} verificará su correcto funcionamiento.

El coste por hora para cada posición se ve resumido en la tabla \ref{costtable}, donde
los sueldos brutos por hora se han obtenido asumiendo 1800 horas anuales. El coste de un
trabajador se obtiene añadiendo un 30\% al sueldo bruto, correspondiente al coste
de la seguridad social. 

\begin{table}[ht]
\begin{center}
    
    \begin{tabular}{ l r r }
     Posición & Sueldo bruto & Coste \\
     \hline
     Jefe de proyecto & 25,28\euro/hora \cite{Projectmanagercost} & 32,86\euro/hora \\
     Científico de datos & 19,60\euro/hora \cite{Datascientistcost} & 25,48\euro/hora \\
     Programador & 16,20\euro/hora \cite{Programmercost} & 21,05\euro/hora \\
     Tester & 11,30\euro/hora \cite{Testercost} & 14,70\euro/hora \\
    \end{tabular}
    \caption{Sueldos brutos y costes por hora. [Elaboración propia]}\label{costtable}
    
\end{center}
\end{table}

En la tabla \ref{taskroledecompos} se descompone el tiempo de cada tarea en horas de desempeño por
cada uno de los roles y se estima el coste económico según una suma ponderada del
tiempo requerido de cada rol y su coste hora visto en la tabla \ref{costtable}.


\newgeometry{a4paper,left=1in,right=1in,top=1in,bottom=1in,nohead}
\begin{landscape}
    \begin{table}
    \centering
    \begin{tabular}{  l  l  r  r  r  r  r  r  }
        Id. & Tarea & J. de proyecto & C. de datos & Programador & Tester & \textbf{Total} & \textbf{Coste} \\
        \hline
        \hyperref[T01]{T01} & Contextualizar y definir el alcance & 15h & 0h & 0h & 0h & 15h & 492,90\euro\\
        \hline
        \hyperref[T02]{T02} & Planificar el trabajo & 5h & 0h & 0h & 0h & 5h & 164,30\euro\\
        \hline
        \hyperref[T03]{T03} & Estimar el presupuesto y analizar la sostenibilidad & 10h & 0h & 0h & 0h & 10h & 328,60\euro\\
        \hline
        \hyperref[T04]{T04} & Reuniones con el director & 5h & 0h & 0h & 0h & 5h & 164,30\euro\\
        \hline
        \hyperref[T05]{T05} & Revisión y edición del documento final & 2h & 1h & 1h & 1h & 5h & 126,95\euro\\
        \hline
        \hyperref[T06]{T06} & Investigar la arquitectura transformer & 0h & 15h & 0h & 0h & 15h & 382,20\euro\\
        \hline
        \hyperref[T07]{T07} & Investigar \textit{fine-tuning} y \textit{transfer learning} & 0h & 10h & 0h & 0h & 10h & 254,80\euro\\
        \hline
        \hyperref[T08]{T08} & Aprender sobre el tratamiento de datos en NLP & 0h & 10h & 0h & 0h & 10h & 254,80\euro\\
        \hline
        \hyperref[T09]{T09} & Estudiar y seleccionar las métricas de valoración & 0h & 5h & 0h & 0h & 5h & 127,40\euro\\
        \hline
        \hyperref[T10]{T10} & Investigar arquitecturas más eficientes de transformer & 0h & 10h & 0h & 0h & 10h & 254,80\euro\\
        \hline
        \hyperref[T11]{T11} & Investigar la técnica \textit{model distillation} & 0h & 5h & 0h & 0h & 5h & 127,40\euro\\
        \hline
        \hyperref[T12]{T12} & Implementar el pipeline de \textit{fine-tunning} & 0h & 10h & 0h & 0h & 10h & 254,80\euro\\
        \hline
        \hyperref[T13]{T13} & \textit{Fine-tunning} de un modelo & 0h & 100h & 0h & 0h & 100h & 2.548,00\euro\\
        \hline
        \hyperref[T14]{T14} & Implementar el pipeline de \textit{transfer learning} & 0h & 10h & 0h & 0h & 10h & 254,80\euro\\
        \hline
        \hyperref[T15]{T15} & \textit{Transfer learning} a un modelo & 0h & 100h & 0h & 0h & 100h & 2.548,00\euro\\
        \hline
        \hyperref[T16]{T16} & Entrenar un modelo más eficiente de transformer & 0h & 120h & 0h & 0h & 120h & 3.057,00\euro\\
        \hline
        \hyperref[T17]{T17} & Implementar una web con placeholders & 0h & 0h & 4h & 1h & 5h & 98,90\euro\\
        \hline
        \hyperref[T18]{T18} & Implementar la traducción de fragmentos de texto & 0h & 0h & 8h & 2h & 10h & 197,80\euro\\
        \hline
        \hyperref[T19]{T19} & Implementar un mecanismo de corrección & 0h & 5h & 8h & 2h & 15h & 325,20\euro\\
        \hline
        \hyperref[T20]{T20} & Implementar traducción de archivos de texto plano & 0h & 0h & 8h & 2h & 10h & 197,80\euro\\
        \hline
        \hyperref[T21]{T21} & Implementar traducción de archivos de texto enriquecido & 0h & 0h & 12h & 3h & 15h & 296,70\euro\\
        \hline
        \hyperref[T22]{T22} & Implementar soporte parcial de traducción de archivos pdf & 0h & 0h & 12h & 3h & 15h & 296,70\euro\\
        \hline
        \hyperref[T23]{T23} & Pulir el proyecto para su uso empresarial \textit{in-house} & 0h & 0h & 4h & 1h & 5h & 98,90\euro\\
        \hline
        \hyperref[T24]{T24} & Redactar la memoria del trabajo & 5h & 40h & 15h & 0h & 60h & 1.499,25\euro\\
        \hline
        \hyperref[T25]{T25} & Preparar la defensa del trabajo & 10h & 0h & 0h & 0h & 10h & 328,60\euro\\
        \hline
    \end{tabular}
    \caption{Desempeño de los distintos roles y coste por tarea. [Elaboración propia]}\label{taskroledecompos}
    \end{table}
\end{landscape}
\restoregeometry

Finalmente, para estimar el coste del personal, se calcula el coste como el producto del desempeño
total en horas de cada rol por su coste hora.

\begin{table}[ht]
    \begin{center}
        \begin{tabular}{ l r r }
         Posición & Tiempo & Coste \\
         \hline
         Jefe de proyecto & 52h & 1.708,72\euro \\
         Científico de datos & 441h & 11.236,68\euro \\
         Programador & 72h & 1.515,60\euro \\
         Tester & 15h & 220,50\euro \\
         \hline
         Total & 580h & 14.681,50\euro
        \end{tabular}
        \caption{Costes de personal del trabajo. [Elaboración propia]}\label{rolcosttable}
    \end{center}
\end{table}


\subsection{Costes genéricos}\label{costesgenericos}
\subsubsection{Internet} El coste de internet es de 90\euro\ mensuales y se utilizará durante los 5 meses de
duración del trabajo con un desempeño de 580 horas en 127 dias.
\begin{figure}[ht]
    \begin{align*}
        Coste_{Internet} = 5 \cdot 90 \cdot \frac{580}{127 \cdot 24} = 85.69\text{\euro}
    \end{align*}
    \caption{Estimación del coste de internet.  [Elaboración propia]}
\end{figure}

\subsubsection{Transporte} En el transcurso del trabajo se efectuarán diversas reuniones con el director. La mayoría de ellas
se realizarán en linea, pero dado que existe la posibilidad de realizar reuniones presenciales, se hará una estimación
a la alza del coste de los transportes asumiendo que todas las reuniones se efectuan presencialmente.

Utilizo transporte público con la tarifa T-Casual de 2 zonas con un coste de 22,40\euro\ cada 10 transportes
\cite{Tcasualcost}. Se consumirán 2 viajes por reunión cada dos semanas durante las 18 semanas que dura el trabajo.
\begin{figure}[H]
    \begin{align*}
        Coste_{Transporte} = \frac{22.40}{10} \cdot 2 \cdot \frac{18}{2} = 40,32\text{\euro}
    \end{align*}
    \caption{Estimación del coste de transporte.  [Elaboración propia]}
\end{figure}

\subsubsection{Electricidad}
El precio medio del kWh durante la primera mitad de octubre del 2021 se situa en 0,3678\euro\ \cite{CosteLuz} y se utilizará como
estimación al alza del precio de la luz de los siguientes meses ya que actualmente el precio del kwH se encuentra
en máximos históricos.

La fuente de alimentación del ordenador que se usará es de 750W pero la mayoría del uso no se realizará bajo carga
máxima y por lo tanto se espera un consumo inferior. Sin embargo a falta de información que me permita acotar
este valor o estimar el consumo de otros periféricos, asumiré que el consumo eléctrico es de 750Wh durante las
580 horas estimadas de duración del proyecto.
\begin{figure}[ht]
    \begin{align*}
        Coste_{Electricidad} = 0.3678 \cdot \frac{750}{1000} \cdot 580 = 160.00\text{\euro}
    \end{align*}
    \caption{Estimación del coste eléctrico.  [Elaboración propia]}
\end{figure}

\subsubsection{Agua}
El coste del agua medio en Catalunya es de \( 2.69\text{\euro}/m^{3} \) \cite{CosteAgua}. Para calcular el coste del agua para
la elaboración del proyecto, asumiré un uso de 123 litros al dia correspondientes al consumo medio de un habitante
en Catalunya \cite{ConsumoAgua}.

Debido a que el proyecto será mi principal ocupación durante los proximo meses y 
la dedicación al proyecto no tiene una relación clara con el consumo de agua, asignare el coste integro del consumo
diario de agua al proyecto para los 127 dias de duración del trabajo \ref{desctasks}.
\begin{figure}[ht]
    \begin{align*}
        Coste_{Agua} = 123 \cdot 127 \cdot \frac{2.69}{1000} = 42.02\text{\euro}
    \end{align*}
    \caption{Estimación del coste del agua.  [Elaboración propia]}
\end{figure}


\begin{figure}[ht]
    \subsubsection{Espacio de trabajo}
    La modalidad del trabajo es de colaboración con la facultad y una condición necesaria para realizar el proyecto
    es la de ser un estudiante matriculado. Como tal, la facultad me facilita espacios de trabajo gratuitos.
    
    Sin embargo, voy a suponer el alquiler de una oficina compartida, en ese caso el coste sería de 50\euro\
    al mes durante los 5 meses de duración del trabajo.
    \begin{align*}
        Coste_{\textit{Espacio de trabajo}} = 50 \cdot 5 = 250\text{\euro}
    \end{align*}
    \caption{Estimación del coste del espacio de trabajo.  [Elaboración propia]}
\end{figure}

\begin{figure}[H]
\subsubsection{Amortización de la maquinaria local}
La elaboración del proyecto requiere de un ordenador \hyperref[R06]{R06} y debe considerarse su 
amortización.
El recurso tiene un precio original de 2.000\euro\ y una expectativa de vida de 6 años, además creo que al final
de su vida tendrá un valor de al menos 200\euro. Asumiré una amortización lineal y debido a que durante los 5 meses
de duración del proyecto dedicaré el recurso casi exclusivamente al proyecto, asigno el coste íntegro al trabajo.
    \begin{align*}
        Coste_{\textit{Maquinaria local}} = \frac{2000-200}{6} \cdot \frac{1}{12} \cdot 5 = 125\text{\euro}
    \end{align*}
    \caption{Estimación de la amortización de la maquinaria local.  [Elaboración propia]}
\end{figure}

\subsubsection{Coste de maquinaria remota}
Los recursos computacionales remotos \hyperref[R07]{R07} són gratuitos gracias a servicios como
Google Colab o Kaggle Kernel.
Se contempla el riesgo de no disponer de recursos suficientes, en cuyo caso se estima el coste en
la sección \ref{maquinariardepago}.

\subsubsection{Software} Todo el software es gratuito y por lo tanto no incurrirá ningún coste.

\subsection{Costes de contingencia}
En el cálculo de los costes se ha decidido realizar estimaciones a la alza siempre que no fuera posible acotar
un coste con exactitud. Tanto el cómputo de horas del personal como los costes genéricos siguen este
criterio además de ser razonablemente detallados.
Es por ello que se ha decidido reservar un ratio de contingencia de solo el 10\% sobre el total.

\subsection{Costes asociados al riesgo}
En la sección \ref{riskmanagement} se detallan los riesgos a considerar durante el transcurso del trabajo y se plantean
diversas soluciones para mitigarlos. Debido a la extensión del proyecto, una buena forma de ajustar la planificación
si aparecen obstáculos, es acotar algunas de las tareas que desde un principio se plantean con la posibilidad de 
ser aplazables para una futura ampliación.

Es por ello que no se incrementarán las horas requeridas para el proyecto, en su lugar se
replanteará la planificación y esto no supone un aumento del coste de personal.
Por otro lado, el riesgo de carecer de recursos computacionales necesarios de forma gratuita requeriria la contratación
de servicios de pago.

\subsubsection{Coste de maquinaria remota de pago}\label{maquinariardepago}
Los recursos computacionales de pago són la solución planteada para resolver una posible insuficiencia de recursos
gratuitos.
Existen muchas opciones, pero los costes exactos són difíciles de calcular porque el uso de ancho de banda,
espacio en el disco, operaciones en el disco, rendimiento y otras características incurren en el precio final en la mayoría
de servicios.

Google, Microsoft, AWS, Linode y otros proveedores ofrecen tarifas segmentadas según
las prestaciones y estas tienen precios parecidos. Asumiré la contratación de un servicio con un coste de 3\euro\ la hora.
Las tareas que requieren de recursos de procesamiento \hyperref[R07]{R07}, suman un total de
130 horas o 80 horas si se decide por aplazar la tarea \hyperref[T16]{T16} como una ampliación futura.
\begin{figure}[H]
    \begin{align*}
        Coste_{\textit{Computo remoto}} = 3 \cdot [200, 320] = [600, 960]\text{\euro}
    \end{align*}
    \caption{Estimación del coste de la maquinaria remota de pago.  [Elaboración propia]}
\end{figure}
Se considerará el coste de 960\euro\ ya que a diferencia de la estimación menor, esta contempla la posibilidad de
realizar el proyecto sin necesidad de ajustes en la planificación.


\begin{table}[H]
    \subsection{Coste total}
    \begin{center}
        \begin{tabular}{ l r }
         Concepto & Importe \\
         \hline
         Costes de personal (CPA) & 14.681,50\euro \\
         Costes genéricos (CG) & 703,03\euro \\
         \textbf{CPA+CG} & 15.384,53\euro \\
         Contingencia del 10\% & 1.538,46\euro \\
         \textbf{CPA+CG+Contingencia} & 16.922,99\euro \\
         Costes asociados al riesgo & 960,00\euro \\ 
         \hline \hline
         \textbf{Total} & 17.882,99\euro
        \end{tabular}
        \caption{Coste total del trabajo. [Elaboración propia]}\label{totalcosttable}
    \end{center}
\end{table}

\section{Control de gestión}\label{controlgestioneco}
Existe la posibilidad de que los cálculos de la gestión económica no sean certeros.
Se han realizado estimaciones a la alza, definido un margen de contingencia y aproximado el coste de los riesgos,
pero es necesario un modelo para medir las desviaciones entre los costes estimados y el real.

Los costes de personal (CPA) són susceptibles a desviaciones debido a la eficiencia del personal, es menos
habitual que se deba a errores en el coste por hora o tarifa. Por otro lado, los costes genéricos (CG) y asociados
al riesgo de este proyecto, són susceptibles a variaciones en las tarifas y menos sensibles a la eficiencia o el
tiempo, ya que en la estimación de estos costes juegan un papel más constante.

Para medir las desviaciones se va a calcular la diferencia entre la estimación
del coste y el coste real para cada uno de los costes.
\begin{figure}[H]
    \begin{align*}
        \textit{Desviación}_{\textit{Coste}} = \textit{Coste estimado} - \textit{Coste real}
    \end{align*}
    \caption{Desviación de la estimación de los costes económicos.  [Elaboración propia]}
\end{figure}



\newgeometry{a4paper,left=1in,right=1in,top=1in,bottom=1in,nohead}
\begin{landscape}
    \begin{table}
        \section{Desviaciones del presupuesto inicial}
        \subsection{Desglose de desviaciones}
        \subsubsection{Desviación del coste de personal}
    \centering
    \begin{tabular}{  l  l  r  r  r  r  r  r  }
        Id. & Tarea & J. de proyecto & C. de datos & Programador & Tester & \textbf{Total} & \textbf{Coste} \\
        \hline
        \hyperref[T01]{T01} & Contextualizar y definir el alcance & 15h & 0h & 0h & 0h & 15h & 492,90\euro\\
        \hline
        \hyperref[T02]{T02} & Planificar el trabajo & 5h & 0h & 0h & 0h & 5h & 164,30\euro\\
        \hline
        \hyperref[T03]{T03} & Estimar el presupuesto y analizar la sostenibilidad & 10h & 0h & 0h & 0h & 10h & 328,60\euro\\
        \hline
        \hyperref[T04]{T04} & Reuniones con el director & 5h & 0h & 0h & 0h & 5h & 164,30\euro\\
        \hline
        \hyperref[T05]{T05} & Revisión y edición del documento final & 2h & 1h & 1h & 1h & 5h & 126,95\euro\\
        \hline
        \hyperref[T06]{T06} & Investigar la arquitectura transformer & 0h & \textbf{30h} & 0h & 0h & \textbf{30h} & \textbf{764,40\euro}\\
        \hline
        \hyperref[T07]{T07} & Investigar \textit{fine-tuning} y \textit{transfer learning} & 0h & 10h & 0h & 0h & 10h & 254,80\euro\\
        \hline
        \hyperref[T08]{T08} & Aprender sobre el tratamiento de datos en NLP & 0h & \textbf{20h} & 0h & 0h & \textbf{20h} & \textbf{509,60\euro}\\
        \hline
        \hyperref[T09]{T09} & Estudiar y seleccionar las métricas de valoración & 0h & \textbf{10h} & 0h & 0h & \textbf{10h} & \textbf{254,80\euro}\\
        \hline
        \hyperref[T10]{T10} & Investigar arquitecturas más eficientes de transformer & 0h & 10h & 0h & 0h & 10h & 254,80\euro\\
        \hline
        \hyperref[T11]{T11} & Investigar la técnica \textit{model distillation} & 0h & 5h & 0h & 0h & 5h & 127,40\euro\\
        \hline
        \hyperref[T12]{T12} & Implementar el pipeline de \textit{fine-tunning} & 0h & 10h & 0h & 0h & 10h & 254,80\euro\\
        \hline
        \hyperref[T13]{T13} & \textit{Fine-tunning} de un modelo & 0h & \textbf{115h} & 0h & 0h & \textbf{115h} & \textbf{2.930,20\euro}\\
        \hline
        \hyperref[T14]{T14} & Implementar el pipeline de \textit{transfer learning} & 0h & 10h & 0h & 0h & 10h & 254,80\euro\\
        \hline
        \hyperref[T15]{T15} & \textit{Transfer learning} a un modelo & 0h & \textbf{115h} & 0h & 0h & \textbf{115h} & \textbf{2.930,20\euro}\\
        \hline
        \hyperref[T16]{T16} & Entrenar un modelo más eficiente de transformer & 0h & \textbf{0h} & 0h & 0h & \textbf{0h} & \textbf{0,00\euro}\\
        \hline
        \hyperref[T17]{T17} & Implementar una web con placeholders & 0h & 0h & 4h & 1h & 5h & 98,90\euro\\
        \hline
        \hyperref[T18]{T18} & Implementar la traducción de fragmentos de texto & 0h & 0h & 8h & 2h & 10h & 197,80\euro\\
        \hline
        \hyperref[T19]{T19} & Implementar un mecanismo de corrección & 0h & 5h & 8h & 2h & 15h & 325,20\euro\\
        \hline
        \hyperref[T20]{T20} & Implementar traducción de archivos de texto plano & 0h & 0h & 8h & 2h & 10h & 197,80\euro\\
        \hline
        \hyperref[T21]{T21} & Implementar traducción de archivos de texto enriquecido & 0h & 0h & 12h & 3h & 15h & 296,70\euro\\
        \hline
        \hyperref[T22]{T22} & Implementar soporte parcial de traducción de archivos pdf & 0h & 0h & \textbf{18h} & 3h & \textbf{21h} & \textbf{423,00\euro}\\
        \hline
        \hyperref[T23]{T23} & Pulir el proyecto para su uso empresarial \textit{in-house} & 0h & 0h & 4h & 1h & 5h & 98,90\euro\\
        \hline
        \hyperref[T24]{T24} & Redactar la memoria del trabajo & 5h & \textbf{50h} & \textbf{20h} & 0h & \textbf{75h} & \textbf{1.859,30\euro}\\
        \hline
        \hyperref[T25]{T25} & Preparar la defensa del trabajo & 10h & 0h & 0h & 0h & 10h & 328,60\euro\\
        \hline
    \end{tabular}
    \caption{Desempeño final de los distintos roles y coste por tarea. [Elaboración propia]}\label{taskroledecomposfinal}
    \end{table}
\end{landscape}
\restoregeometry

En la sección \ref{desviaplan} se argumenta la decisión de descartar la tarea \hyperref[T16]{T16} y la necesidad de ajustar el tiempo dedicado de algunas tareas. En la figura \ref{taskroledecomposfinal} se plasman las desviaciones.

Estas desviaciones generan costes de personal, concretamente la eliminación de la tarea \hyperref[T16]{T16} supone un decremento de 3.057,00\euro. El resto de modificaciones suponen un aumento en el desempeño que implica un incremento en el coste.
Se calcula una desviación total de personal de 1042,45\euro.
\subsubsection{Desviación de los costes genéricos}
Los costes de internet y electricidad se calculan según la consumición estimada durante la realización del trabajo, es por ello que
deben ajustarse al recuento final de horas según la planificación final. Se estima que con una duración de 541 horas, la desviación del coste
de internet es de 5,82\euro\ y el de electricidad es de 10,77\euro\ usando las fórmulas definidas en el apartado \ref{costesgenericos}.
El coste del transporte, valorado en 40,32\euro, debe descartarse ya que todas las reuniones con el director se han realizado telemáticamente.
Se calcula una desviación total de costes genéricos de 56,91\euro.
\subsubsection{Desviación del coste de contingencia}
El 10\% sobre el total del coste de personal (CPA) y genéricos (CG), ha resultado ser innecesario ya que los costes reales són inferiores a la estimación a la alza. La desviación del coste de contingencia es de 1.538,46\euro.
\subsubsection{Desviación del coste asociado al riesgo}
Durante la elaboración del presupuesto se estimó un coste de 960,00\euro\ reservado para la contratación de maquinaria remota en el entrenamiento de los modelos de traducción. Afortunadamente el proyecto no ha necesitado procesamiento remoto y no se ha usado este fondo.

\subsection{Desviación total}
La desviación presupuestaria se calcula como la diferencia entre la estimación y el coste real tal como se definió en el apartado \ref{controlgestioneco}.
\begin{table}[H]
    \begin{center}
        \begin{tabular}{ l r r r}
         Concepto & Estimación & Real & Desviación \\
         \hline
         Costes de personal (CPA) & 14.681,50\euro & 13.639,05\euro & 1042,45\euro \\
         Costes genéricos (CG) & 703,03\euro & 646,12\euro & 56,91\euro \\
         \textbf{CPA+CG} & 15.384,53\euro & 14.285,17\euro & 1.099,36\euro \\
         Contingencia del 10\% & 1.538,46\euro & 0,00\euro & 1.538,46\euro \\
         \textbf{CPA+CG+Contingencia} & 16.922,99\euro & 14.285,17\euro & 2.637,82\euro \\
         Costes asociados al riesgo & 960,00\euro & 0,00\euro & 960,00\euro \\ 
         \hline \hline
         \textbf{Total} & 17.882,99\euro & 14.285,17\euro & 3.597,82\euro
        \end{tabular}
        \caption{Desviación total del presupuesto. [Elaboración propia]}\label{totaldesvictable}
    \end{center}
\end{table}

En el cuadro \ref{totaldesvictable} se aprecia una desviación positiva del coste de elaboración del trabajo debido a la decisión de estimar los costes al alza además de reservar fondos de contingencia y asociados al riesgo. Se ha comprobado que los costes reales són inferiores a la estimación a la alza tal como se planteó.