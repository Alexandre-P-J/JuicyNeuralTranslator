\section{Justificación y vía de resolución}
La traducción neuronal basada en la arquitectura transformer consume una gran cantidad de recursos,
por ese motivo, el modelo de distribución cliente-servidor es ideal para desacoplar el cálculo del
servicio de traducción. Ejecutando los modelos de traducción en un servidor optimizado acordemente,
se elimina la necesidad del usuario de disponer de una máquina de altas prestaciones. Es por ello
en el caso de implementar una herramienta nueva, o adaptar una solución existente, sería adecuado
que funcionara vía web.

Existen diversos servicios web que proporcionan traducción automática de calidad, pero las soluciones
existentes tienen un coste economico elevado, difícil de estimar y no escalable a un grado
empresarial con necesidad de traducir mucho contenido \cite{GoogleTranslatePricing,AWSTranslatePricing,MicrosoftTranslatePricing}.
Además no permiten el ajuste de los modelos para traducir contextos o registros del lenguaje
específicos, una funcionalidad necesaria en muchos escenarios reales como por ejemplo el de la oficina
europea de patentes \cite{PatentOffice2020Jul}.

Por otra parte, en los últimos años han aparecido iniciativas como OpusMT
\cite{HelsinkiTiedemannThottingal2020} con el objetivo de distribuir corpus de texto y modelos de
traducción pre-entrenados. Estos modelos van orientados a investigadores en el campo del
\textit{machine learning} y serán de vital importancia para la elaboración del proyecto.

Es preciso implementar una solución nueva dado que no esta en mi conocimiento ninguna
herramienta abierta que permita la traducción de texto y archivos usable a nivel empresarial
y con la capacidad de ajustar y añadir nuevos modelos. Pero también se utilizarán modelos
pre-entrenados ya que serán un buen punto de partida para aplicar técnicas de
\textit{fine-tuning} y \textit{transfer learning}.

