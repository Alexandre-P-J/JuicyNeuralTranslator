\section{Contexto}
Este documento es un trabajo del grado en ingeniería informática en la especialidad
de computación. Se realiza en la facultad de informática de la universidad politécnica
de Barcelona bajo la dirección de Javier Béjar Alonso, doctor en ingeniería informática.

\subsection{Introducción}\label{intro}
En las últimas décadas ha habido un esfuerzo continuo en la investigación de modelos para la traducción
automática.

Los métodos clásicos consistian en sistemas de reglas definidas manualmente que se
basaban en el conocimiento lingüístico, pero su efectividad era muy limitada dada la complejidad del lenguaje
natural.

Con la aparición de corpus bilingües de gran tamaño, los modelos estadísticos se popularizaron
porque a diferencia de los métodos basados en reglas, estos eran capaces de aprender
estructuras latentes: como alineamientos de palabras y frases
\cite{HistoryBrown1990ASA, HistoryKoehn2003Jan}. Pero dada su incapacidad para
modelar dependencias entre palabras muy distanciadas, la calidad de traducción no era ideal.
Los modelos basados en redes neuronales \cite{Historykalchbrenner-blunsom-2013, HistoryCho2014Jun,
HistorySutskever2014Sep, HistoryBahdanau2014Sep} reemplazaron los métodos anteriores de
traducción automática gracias al avance en procesamiento del lenguaje natural el campo
del \textit{deep learning} \cite{HistoryTan2020Dec}.

Las mejores arquitecturas de traducción neural hacían uso de modelos recurrentes complejos como
\textit{LSTM} \cite{HistoryHochreiter1997} y \textit{GRU} \cite{HistoryChung2014Dec} o
convolucionales \cite{HistoryGehring2016Nov}. Además tenían en común el uso
de \textit{encoder-decoder} con algún tipo de mecanismo de atención \cite{HistoryBahdanau2014Sep} y
eso mismo es lo que inspiró la arquitectura \textit{Transformer} \cite{Vaswani2017Jun},
cuya arquitectura y variantes representan el estado del arte en traducción automática.

Este trabajo tiene como objetivo la búsqueda, investigación y el desarrollo de modelos
basados en transformers para la implementación de una herramienta
empresarial de traducción de texto orientada a usuarios ajenos al \textit{machine learning}.

\subsection{Conceptos y definiciones preliminares}
El trabajo se va a centrar en la arquitectura transformer \cite{Vaswani2017Jun} y variantes 
para la tarea de traducción automática. Sin embargo, el estudio de las arquitecturas
forma parte de la porción teórica del proyecto y se explicarán segun sean empleadas durante el trabajo.
Por otra parte es importante definir los siguientes conceptos:

\subsubsection*{Modelos secuencia-secuencia}
Los modelos secuencia-secuencia son aquellos entrenados para
resolver tareas que requieren la entrada de una secuencia y el retorno de
otra secuencia.
En el caso de la traducción automática la secuencia de entrada podria ser
un fragmento de texto escrito en inglés y la secuencia de salida el texto
traducido al español.

\subsubsection*{Modelos encoder-decoder}
La arquitectura encoder-decoder es un patrón de alto nivel y tal como su nombre indica,
utiliza uno o más encoders y decoders.
Un encoder tiene la tarea de transformar datos a un espacio latente
mientras que un decoder hace la operación inversa. Al tratarse de un concepto de alto nivel, la
implementación depende del propio modelo.

\subsubsection*{Mecanismos de atención}
Los mecanismos de atención son una forma de codificar la importancia y relaciones de uno o más
datos de la entrada. Un ejemplo seria el de las palabras en una oración: cada palabra tiene una
importancia y unas relaciones con el resto de palabras de la oración.
En el caso concreto del transformer original \cite{Vaswani2017Jun} el mecanismo de atención
calcula numericamente la relación entre cada par de palabras, pero existen diversos métodos.

\subsubsection*{\textit{Fine-tuning y Transfer learning}}
\textit{Fine-tuning} consiste en re-entrenar (posiblemente con datos distintos) un modelo que ya ha
sido entrenado anteriormente. Por otro lado, \textit{transfer learning} o transferencia de conocimiento
consiste en re-entrenar un modelo para realizar una tarea distinta a la que fue entrenado anteriormente,
con la intención de que el conocimiento del primer entrenamiento pueda ser aplicado en la nueva tarea.

El proyecto pretende usar \textit{fine-tuning} en un modelo de traducción para ajustar el registro
o contexto de sus traducciones. También se aplicará \textit{transfer learning} para intentar que un modelo
capaz de traducir dos lenguajes aprenda a traducir lenguajes distintos, transfiriendo conocimientos de los
lenguajes con los que fue entrenado, a los nuevos.

\subsubsection*{\textit{Model distillation}}
Esta técnica consiste en la transferencia de conocimiento de un modelo complejo (o un conjunto de modelos)
a un nuevo modelo.
Es una técnica interesante porque el nuevo modelo puede ser más pequeño, simple y eficiente,
además reduce el tiempo de entrenamiento considerablemente.
Si el tiempo lo permite, se hará \textit{model distillation} de un modelo pre-entrenado de traducción
hacia una arquitectura más eficiente.




\subsection{Identificación del problema}\label{problem}
La traducción automática tiene usos empresariales y particulares, sin embargo las opciones
actuales tienen uno o más de los problemas mencionados a continuación:

\subsubsection*{Coste económico}
Google, Amazon, Microsoft y otras compañías ofrecen servicios de calidad pero el coste
suele ser elevado y difícil de calcular
\cite{GoogleTranslatePricing,AWSTranslatePricing,MicrosoftTranslatePricing}.

\subsubsection*{Coste computacional}
Las herramientas de traducción neuronal consumen muchos recursos.

\subsubsection*{Inmutabilidad}
Algunas empresas tienen necesidades de traducción que se limitan a registros de
lenguaje y contextos específicos, pero la mayoría de servicios usan modelos que no
permiten ningún tipo de ajuste.

\subsubsection*{Accesibilidad}
Existen modelos de calidad y con licencias permisivas que no contemplan la posibilidad de
ser usados por gente ajena al \textit{machine learning}.


\subsection{Actores implicados}
El proyecto tiene diversos actores implicados que pueden agruparse en dos grupos dependiendo
del tipo de interacción con el proyecto.

Los actores que interactuan de forma directa con el proyecto son el director y el investigador.
Javier Béjar Alonso es el director del proyecto y guiará al investigador a correcto desarrollo
del trabajo. El investigador Alexandre Pérez Josende será responsable de planificar, investigar,
desarrollar, experimentar y documentar el proyecto.

Los actores que no interactuan con el trabajo pero se benefician, son compañías y particulares
con necesidad de un sistema de traducción automático. Además los investigadores, que también reciben
acceso a este trabajo, podrán hacer uso de la información y conclusiones que deriven del proyecto.