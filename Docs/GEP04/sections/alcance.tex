\section{Alcance}
\subsection{Objetivos y subobjetivos}\label{objectives}
El objetivo principal es desarrollar una herramienta empresarial para la traducción automática
de texto.
Se pretende que la herramienta solvente los problemas mencionados en la sección \ref{problem}.

Para lograr este objetivo se ha subdividido el proyecto en los siguientes subobjetivos:

\subsubsection*{Parte teórica}
    \begin{itemize}
        \item Investigar la arquitectura transformer.
        \item Investigar sobre las posibilidades del \textit{fine-tuning} y \textit{transfer learning}
            en la arquitectura transformer para tareas de traducción.
        \item Investigar distintos métodos de adquisición y tratamiento de los datos en el contexto del
            procesamiento del lenguaje natural.
        \item Investigar sobre las métricas que se usarán para estimar la bondad de los modelos de traducción.
        \item Investigar arquitecturas más recientes de transformer y la posibilidad de entrenar
        un modelo más eficiente. Quizás mediante \textit{model distillation}.
    \end{itemize}
\subsubsection*{Parte práctica}
    \begin{itemize}
        \item Elegir uno o más modelos pre-entrenados y programar el \textit{pipeline} necesario para
        realizar \textit{fine-tuning} y \textit{transfer learning}.
        \item Realizar \textit{fine-tuning} de un modelo para ajustar el registro o contexto
        lingüístico de la traducción.
        \item Realizar \textit{transfer learning} en un modelo de traducción para que traduzca
        una lengua distinta.
        \item Programar una aplicación web para la traducción automática de texto que permita
        usar modelos pre-entrenados y modelos de curación propia.
            \begin{itemize}
                \item Implementar una interfaz para la traducción de fragmentos de texto.
                \item Implementar un mecanismo de corrección y mejora.
                \item Implementar un método de traducción de documentos.
                \item Preparar y documentar el proyecto para ser usado por empresas e individuos con
                    los recursos necesarios.
            \end{itemize}
        \item Si el tiempo y los resultados del proyecto lo permiten, entrenar un modelo más eficiente
            de transformer.
    \end{itemize}



\subsection{Requisitos}
Se requieren los requisitos listados a continuación para garantizar la calidad del proyecto.

\subsubsection{Requisitos funcionales}
\begin{itemize}
    \item La interfaz de traducción debe dotar a usuarios ajenos al \textit{machine learning} la capacidad
    de traducir texto usando los modelos implementados durante el proyecto.
    \item La herramienta traducirá fragmentos de texto y archivos proporcionados por el usuario. 
    El catálogo de formatos de archivo soportados estará sujeto al tiempo disponible para
    dicha funcionalidad.
    \item Debe implementarse un mecanismo de corrección que permita el refinamiento de los modelos
    de traducción.
\end{itemize}

\subsubsection{Requisitos no funcionales}
\begin{itemize}
    \item La estimación del rendimiento de los modelos debe realizarse en base a una o más métricas definidas
    a priori.
    \item El origen de los datos usados y su método de obtención debe tenerse en cuenta
    para garantizar la separación de las particiones de entrenamiento, validación y estimación del rendimiento.
    \item Se requiere hardware de alto rendimiento y el tiempo necesario para el entrenamiento
    de los modelos.
    \item La distribución del proyecto debe facilitar su implementación en entornos empresariales.
    \item Se deben usar buenas prácticas de programación, con un estilo legible y la menor
        complejidad posible.
\end{itemize}

\subsection{Obstáculos y riesgos}
El proyecto presenta diversos riesgos que podrian interferir en su desarrollo, asimismo, varios obstáculos
potenciales deben tenerse en consideración en el transcurso del proyecto.
\subsubsection{Fecha límite del trabajo}
Hay una fecha límite para la entrega del proyecto que condiciona su desarrollo, por lo que
es necesario disponer de una buena planificación para terminar el proyecto a tiempo.
\subsubsection{Recursos computacionales}
Los modelos transformer y sus diversas variantes consumen una gran cantidad de memoria de video.
Suponiendo que se dispone de la memoria necesaria, el tiempo de entrenamiento para la tarea de traducción
puede tardar dias hasta converger. Es por ello que de se va a priorizar el \textit{fine-tuning} y
\textit{transfer learning} de modelos pre-entrenados para reducir el tiempo de procesamiento.
La implementación de un modelo alternativo de transformer podria quedar como una tarea futura.
\subsubsection{Inexperiencia en el campo de estudio}
Nunca he experimentado con modelos en el campo procesamiento del lenguaje natural
y mi experiencia general con modelos de deep learning es limitada. El trabajo tiene un fuerte componente
de investigación y la lectura de papers consume tiempo que se debe tener en cuenta en la
planificación.
Del mismo modo, el director del proyecto será de gran ayuda ya que sus recomendaciones me guiarán en el
transcurso del proyecto.
\subsubsection{Bugs y otras dificultades de programación}
Un bug en el código podria afectar de manera inesperada al resultado. Es por ello
que se programará código legible de la forma más simple posible, optando por usar librerias testeadas siempre
que sea viable. De esta manera se reducirá la carga de trabajo y las posibilidades de cometer errores.
